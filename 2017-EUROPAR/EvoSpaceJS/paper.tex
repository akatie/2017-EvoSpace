
\documentclass{llncs}
\usepackage{graphicx}        % standard LaTeX graphics tool
                             % when including figure files
\usepackage{url}
%%%%%%%%%%%%%%%%%%%%%%%%%%%%%%%%%%%%%%%%%%%%%%%%%%%%%%%%%%%%%%%%%%%%%%%%%%%%%%%%%%%%%%%%%

\begin{document}
\sloppy

\title{Pool-Based Asynchronous Grey Wolf Optimizer}

\author{Mario Garc\'ia-Valdez\inst{1} \and Luis Rodriguez\inst{1} \and Juan J. Merelo Guerv\'os\inst{2}}

\institute{Instituto Tecnol\'ogico de Tijuana, Tijuana BC, Mexico
\and
Universidad de Granada, Granada, Spain
\email{mario@tectijuana.edu.mx}\\
\email{renemarquezvalenzuela@gmail.com}\\
\email{jmerelo@geneura.ugr.es}}

\maketitle

\begin{abstract}
 

    Using the EvoSpace model, in this paper an asynchronous distributed version of 
    the Grey Wolf Optimizer (GWO) algorithm is presented. The EvoSpace model is
    built around a central repository or population store, incorporating some of 
    the main principles of the tuple-space model. Remote workers,
	take random samples of the population to perform on them the basic evolutionary
	processes, once the work is done, the modified sample is pushed back to the 
	central population. The GWO distributed version has been benchmarked using 
	four well-known test functions using a two, three, six and twelve worker configurations. 
	Results show that the parallel version has competitive results when compared 
	to a centralized version.   

	


	

\keywords{Distributed Evolutionary Algorithms, Cloud Computing}


\keywords{Distributed Evolutionary Algorithms, Volunteer Computing}
\end{abstract}

\end{document}
