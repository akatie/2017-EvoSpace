
\documentclass{llncs}
\usepackage{graphicx}        % standard LaTeX graphics tool
                             % when including figure files
\usepackage{url}
%%%%%%%%%%%%%%%%%%%%%%%%%%%%%%%%%%%%%%%%%%%%%%%%%%%%%%%%%%%%%%%%%%%%%%%%%%%%%%%%%%%%%%%%%

\begin{document}
\sloppy

\title{Random Selection of Parameters in Asynchronous Pool-Based Evolutionary Algorithms}

\author{Mario Garc\'ia-Valdez\inst{1} \and Ren\'e M\'arquez\inst{1} \and Juan J. Merelo Guerv\'os\inst{2} \and  Leonardo Trujillo \inst{1}}

\institute{Instituto Tecnol\'ogico de Tijuana, Tijuana BC, Mexico
\and
Universidad de Granada, Granada, Spain
\email{mario@tectijuana.edu.mx}\\
\email{renemarquezvalenzuela@gmail.com}\\
\email{jmerelo@geneura.ugr.es}
\email{leonardo.trujillo@tectijuana.edu.mx}}

\maketitle

\begin{abstract}
Heterogeneous devices or virtual machines have special requirements if
their full performance is going to be actually used by evolutionary
algorithms. Several asynchronous Evolutionary Algorithms (EAs) have
been proposed that distribute the evolutionary process among these
devices; in these algorithms the population is shared between
distributed worker processes which execute the actual evolutionary
process by taking samples of the population, and replacing them in the
population pool evolved individuals. The performance of these EAs
depends in part on the selection of parameters for the EA running in
each worker, which may include sample size, generations, mutation rate
and crossover rate along with the overall configuration. In this paper
we evaluate a strategy proposed by Gong and Fukunaga for the
Island-Model which statically assigns random parameter settings to
each island in a cloud setting. Experiments were conducted in the
Amazon cloud using 2, 6 and 12 virtual machine configurations, with
both homogeneous and heterogeneous random settings using five 
test functions for single-objective optimization (Rastrigin, Griewank, De Jong, Schaffer 
and Ackley) and the one-max binary problem. The results suggest that this approach can yield
performance improvements which are competitive to instances of the
algorithm using workers with control parameters tuned specifically for
the benchmark.

\keywords{Distributed Evolutionary Algorithms, Volunteer Computing,
  Cloud Computing}
\end{abstract}

\section{Experiments}
\subsection{One Max}

\subsection{Benchmark Functions}
\section{Results}



\end{document}
