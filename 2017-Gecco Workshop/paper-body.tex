\section{Introduction}
% Main Ideas:
%Why Paralel
A large body of work exists on the parallelization of EAs,
with techniques using multiple CPU cores, many compute nodes, 
and GPUs \cite{cantu2000efficient,hofmann2013performance}. 
However, asynchronous EAs have started to become common only
recently, in an effort to exploit computing resources available
through different Internet technologies. In this work, we are
interested in those EAs following a pool-based approach,
where a collection of heterogeneous worker processes 
conduct the search by collaborating through a shared 
repository or population pool. We will refer to such algorithms 
as Pool-based EAs or PEAs, and highlight the fact that 
such systems are intrinsically parallel, distributed and asynchronous.

Pool-EAs differ from the closely related island model, 
mainly with regards to the responsibilities assigned to 
the server. In the island model, the server is usually 
responsible for the interaction and synchronization of 
all the populations, while in a Pool-EA the server only 
receives stateless requests from isolated workers 
or clients. In this way, Pool-EAs are capable of an 
ad-hoc collaboration of computing resources. 

The platform presented in this paper is a new implementation 
of the EvoSpace model \cite{GValdez2015} in which 
workers asynchronously interact with the population 
pool by taking samples of the population 
to perform a local evolutionary search on the samples, 
to then return newly evolved solutions back to the pool.

The previous version was implemented using CherryPy a basic HTTP 
server written in Python. This new version uses Node.js an 
event-driven server capable of asynchronous I/O, that is 
running on the JavaSacript V8 engine. Node.js is used 
to optimize throughput and scalability of the server.
Additionaly to the  increased performance this version 
adds new functionality: In the former version workers could only
ask for random samples of a particular size, now clients 
can retrieve objects from the server ordered by a score. 
Designers can use this functionality to implement 
asynchronous versions of the Island Model or to force 
the retrieval of different objects in every request 
resembling a circular queue. Instead of using the JSON-RPC 
protocol the server functionality is now exposed as a RESTful 
Web Service. The server now keeps a log of the work performed 
by workers: The number of evaluations, the best solution in each 
generation (or iteration), parameters and algorithm used among others.
This log can later be used to compare the performance of 
the algorithm against others, for instance against 
algorithms using the COCO (COmparing Continuous Optimisers)
platform \cite{hansen2016coco}.
The aim of the evospace-js software is to provide 
researchers with a high performance platform in which 
they can execute pool-based algorithms using heterogeneous workers. 

The remainder of the paper proceeds as follows. Section \ref{sec:work} 
reviews related work. Afterwards, Section \ref{sec:evo} describes the
proposed EvoSpace implementation, the experimental work is presented in 
Section \ref{sec:experiments}. Finally, a summary and 
concluding remarks are given in Section \ref{sec:conclusions}.


\section{Related Work}
\label{sec:work}
There are two important practical issues faced by many EA systems, namely the size of the parameter 
space and the high computational cost when it is compared with mathematical programming or numerical techniques.
Concerning the latter, one approach to mitigate this issue is to use parallel or 
distributed implementations \cite{cantu-paz:migration-policies,duda2013gpu}.
For instance, Fern\'andez et al. \cite{nc} % articulo Paco, Gustavo y Leo publucado en Natural Computing}
uses the well-known Berkeley Open Infrastructure for Network Computing (BOINC) to distribute EA runs across a
heterogeneous network of volunteer computers using virtual machines. Another recent example is 
found in the FlexGP system developed by Sherry et al. \cite{sherry2012flex}. FlexGP is probably the first large scale GP system 
that runs on the cloud, using an Island model approach and implemented over Amazon EC2 with a 
socket-based client-server architecture.

Another approach to distributed EAs is the so called pool-based architecture. In general, a 
pool-based system employs a central repository (real or virtual) where the evolving population is stored.
Distributed clients interact with the pool, performing some or all of the basic EA processes 
(selection, genetic operators, survival). A representative work of this approach 
is that by Merelo et al. \cite{agajaj} implementing a JavaScript based PEA that distributes 
the evolutionary process over the web, providing the added advantage of not requiring the 
installation of additional software in each computing node.  Other similar cloud-based solutions 
are based on a global queue of tasks and a Map-Reduce implementation which normally handles failures 
by the re-execution of  tasks \cite{fazenda2012,di2013towards,FlexGP}. Using the BOINC 
volunteer platform  Smaoui et al. \cite{FekiNG09} uses work units that consist of a fitness 
evaluation task and multiple replicas  were produced and sent to different clients.

While using a distributed framework can ease the computational cost, it can also exacerbate the first issue mentioned above;
i.e., it increases the size of the algorithm parameter space, which makes parameter tuning a more difficult task.
The issue of optimal parametrization of EAs is a widely studied subject \cite{de2007parameter}, 
with many approaches in literature. For instance, one of the most successful approaches 
is the F-Racing and iterative F-Racing techniques \cite{lopez2011irace}. 
However, while such algorithms can find high performance parametrization, 
they require additional computational effort which can be too expensive in some applications
(even if they are more efficient than an exhaustive search).



\section{evospace-js Implementation}
\label{sec:evo}
The main components of the EvoSpace framework are: the evospace-js 
population container, remote clients called EvoWorkers.
Each of these components are defined in the following subsections.

\subsection{evospace-js}
 \label{sec:evospace}
The evospace-js server provides a collection of REST methods that 
operate over a set of objects $ES$, which can be seen as the 
population. Multiple populations can be created and are 
distinguished by their name. Objects in $ES$ represent 
individuals in the population (these are defined in the next subsection), 
they can be selected, removed and replaced from $ES$ using
a specified set of REST methods. The main interfaces are:
\begin{enumerate}
    \item {\bf population\_name/initialize} 
    This is a {\tt POST} request used to create a new population.
    \item {\bf population\_name/individual} 
    This is a {\tt POST} request used to create and add a new object
    to a population. The object is defined in a JSON format, 
    and there is no restriction on its structure, only 
    the following properties are required: ``id'' this is an 
    integer and is generated if not present, ``fitness'' also defined 
    as a JSON object, the structure been specific to each application, 
    and finally a ``chromosome'' property again defined as
    a JavaScript object giving the internal representation of 
    the solution, by default a it defined as list of objects. 
    There is also an optional integer property called 
    ``score'' used when objects are going to be retieved in a certain order.
    \item {\bf population\_name/sample/n}
    This is a {\tt GET}  request used to take from the population a 
    sample of {\bf n} objects. These objects are removed from the population and are no longer available
    to other requests until and only if they are put back. Objects can be returned to the population 
    either by a {\tt PUT} sample request called from the same client or by a Respawn request. The reason for 
    this is to avoid concurrently write conflicts and duplication of work.
    \item {\bf population\_name/sample}
    This is a {\tt POST} request used to put back a sample to the population.
    The new sample is sent in the request body as a JSON object. If the client created new objects or 
    changed their original state, these objects replace the originals. 
    \item {\bf population\_name/respawn}
    This is a {\tt POST} request used to put back {\bf n} samples to their 
    original state. The number of samples is sent in the request body. 
\end{enumerate}
There are other secondary methods used to: select all objects in a 
population, select objects with scores with in a range, read the 
top {\tt n} objects according to a score and read the number of
objects currently on the population.      

The above methods were implemented first as JavaScript library 
with two classes: {\tt Individual} and {\tt Population} depicted 
in Figure  with calls to the Redis memory store through the {\bf ioredis} 
asynchronous library. 

In order to expose the library as a REST Web service an HTTP 
interface is implemented using the Express HTTP framework. 
An optional dashboard type application, can be used to inspect 
the populations currently available on the server.

  

\subsection{EvoWorkers}
\label{sec:evoworkers}


\section{Experiments}
 \label{sec:experiments}



\section{Conclusions and Further Work}
\label{sec:conclusions}

% To be written

Future lines of work will focus on using other EA or meta-heuristic techniques, 
such as genetic programming or particle swarm optimization for having 
workers that are heterogeneous in more than one sense. RPSS could be
used in those cases where each algorithm has different sets of
parameters, but also to randomly select the technique used in each
node. Another interesting line of work is the dynamic adaptation of
parameters by measuring the diversity of each worker or returned
sample. This could be specially useful in cases where the random
parametrization technique seems to achieve bad results. 

\begin{acks}
This work has been supported in part by:  Ministerio espa\~{n}ol de
Econom\'{\i}a y Competitividad under project TIN2014-56494-C4-3-P
(UGR-EPHEMECH).
\end{acks}
